\selectlanguage{spanish}
\renewcommand{\headline}{\section*{{\LARGE{}Penta$\cdot$game} - Spanish}}
\renewcommand{\tocent}{Spanish}
\renewcommand{\translator}{Carlos Ayon}
\renewcommand{\general}{
Un juego que puede ser explicado en un minuto, pero se mantiene fascinante 
por años.

Dos jugadores juegan durante 20 minutos. Tres o cuatro jugadores 
pueden tardar hasta 90 minutos en terminar un juego. 

No hay dados involucrados. Es apropiado para toda persona 
mayor a los 5 años. Un juego creado por Jan \noun{Suchanek.}

El juego incluye: 4$\times$5 figuras pintadas a mano, 5 bloques negros y
5 grises, la tabla de juego.
}

\renewcommand{\choosex}{
\subsubsection*{Elije tus figuras}
Cada jugador tiene cinco figuras. Se usa un grupo de figuras por jugador. 
Un equipo tiene cabello plateado, un equipo tiene cabello negro, 
uno tiene cabello dorado y uno es calvo

Los jugadores deberán jugar en el equipo al que mas se parezcan.

Se tienen figuras azules, rojas, blancas, verdes y amarillas. 

Comienzan en las 5 esquinas de la tabla de juego, en su respectivo color.

Su objetivo es alcanzar los espacios grandes de la mitad de la tabla.  
}

\renewcommand{\setup}{
\subsubsection*{Preparación}
Coloque sus figuras en las esquinas grandes a la orilla de la tabla 
en los colores correspondientes al color de la pieza: Su figura blanca
en la esquina blanca al borde de la tabla, su figura azul en la esquina azul, etc. 

Coloque los bloques negros en las cinco intersecciones al medio de la tabla.

Coloque los bloques grises en el centro, para utilizar después.
}

\renewcommand{\objective}{
\subsubsection*{Objetivo}
Las figuras blancas quieren ir a la intersección blanca en el centro, 
las azules quieren ir a la intersección azul, etc. El destino es siempre 
el sitio grande, coloreado, que se encuentra de lado opuesto al que se
encuentra la figura al comenzar la partida. 
White figures want to go to the white crossing in the centre, blue
ones to the blue crossing, etc. The destination is always the big
coloured stop in the middle opposite the starting point.

Se el primero en mover \emph{tres} piezas a su destino para ganar.
}

\renewcommand{\rules}{
\subsection*{Reglas}
Mueva cualquiera de sus figuras en la estrella o el borde en cualquier dirección, a la distancia
que prefiera. 

Puede dar la vuelta en cualquier esquina o intersección libres, sin detenerse.  

Nunca salte\textemdash sobre los bloques o cualquier otra figura. 

\medskip

Pero puede moverse \emph{a} un espacio que esta ocupado.

\medskip

Si al hacerlo toma el lugar de un bloque negro, coloque este bloque en un espacio 
libre de su elección

Si se mueve al lugar de otra figura, intercambie posiciones con esta. 
 
\medskip

Puede intercambiar posiciones con dos de sus propias figuras. 

Si se mueve a un sitio con múltiples figuras, elija una de ellas con la cual 
intercambiar posiciones. 
 
\medskip 
 
No se permite realizar el mismo movimiento dos veces, en sucesión.  

\medskip 

Al alcanzar su destino, la figura es removida de la tabla de juego y se coloca en el centro de la tabla.
Después se toma un bloque gris y se coloca en un espacio libre de su elección.

Si algún jugador toma el sitio del bloque gris, dicho bloque es removido de la tabla de juego. 

\medskip

Quien mueva \emph{tres }de sus figuras fuera de la tabla, ha ganado.

\medskip

El habla excesiva es motivo de descalificación.
}
